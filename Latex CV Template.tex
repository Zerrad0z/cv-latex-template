% Contenu générique pour un CV en français
\documentclass[a4paper,11pt]{article}
\usepackage[utf8]{inputenc}
\usepackage[top=1cm,bottom=1cm,left=1cm,right=1cm]{geometry}
\usepackage{titlesec}
\usepackage{enumitem}
\usepackage{xcolor}
\usepackage{fontawesome5}
\usepackage{tikz}
\usepackage{calc}
\usepackage{graphicx} % Package to inclure images

\definecolor{primarycolor}{RGB}{0,91,150}
\definecolor{linecolor}{RGB}{200,200,200}

\titleformat{\section}{\color{primarycolor}\Large\bfseries}{\thesection}{1em}{}[\titlerule]
\titlespacing*{\section}{0pt}{*0.5}{*0.5}
\newenvironment{smallerfont}{\small}{}
\newcommand{\cvitem}[2]{\textbf{#1}#2\\}

% Define column widths
\newlength{\leftcolwidth}
\newlength{\rightcolwidth}
\setlength{\leftcolwidth}{0.35\textwidth}
\setlength{\rightcolwidth}{0.62\textwidth}
\pagenumbering{gobble}

\newcommand{\sectiontitle}[1]{%
  \vspace{0.5em}
  \textcolor{primarycolor}{\Large\textbf{#1}}%
  \par\vspace{-0.3em}%
  \textcolor{primarycolor}{\rule{\linewidth}{0.4pt}}%
  \vspace{0.3em}%
}

\begin{document}

% En-tête avec le nom et titre générique
\vspace*{-0.3cm}
\noindent\begin{minipage}[t]{0.75\textwidth}
\color{primarycolor}\Huge\textbf{NOM PRÉNOM}\\[0.1em]
\large Titre Professionnel (ex. Développeur FullStack, Chef de Projet IT)
\end{minipage}%

% Ajout d'une photo (optionnelle)
\begin{tikzpicture}[remember picture, overlay]
\node[anchor=north east, xshift=-1.5cm, yshift=0cm] at (current page.north east) {\includegraphics[width=103.5pt, height=105pt]{profile.png}};
\end{tikzpicture}

\noindent\begin{minipage}[t]{\leftcolwidth}
\sectiontitle{\faIcon{address-book} CONTACT}
\cvitem{\faIcon{phone}}{ +33 6 12 34 56 78}
\cvitem{\faIcon{envelope}}{ email@exemple.com}
\cvitem{\faIcon{linkedin}}{ linkedin.com/in/exemple}
\cvitem{\faIcon{github}}{ github.com/nom-utilisateur}

\sectiontitle{\faIcon{graduation-cap} FORMATION}
\cvitem{[20XX-20XX]}{\textbf{Diplôme/Master en Domaine}}
\cvitem{}{Nom de l'établissement}
\cvitem{[20XX-20XX]}{\textbf{Licence/Autre Diplôme}}
\cvitem{}{Nom de l'établissement}

\sectiontitle{\faIcon{certificate} CERTIFICATIONS}
\cvitem{}{\textbf{- Titre de certification 1}}
\cvitem{}{\textbf{- Titre de certification 2}}

\sectiontitle{\faIcon{cogs} COMPÉTENCES}
\cvitem{Programmation :}{Langages ici (ex. Java, Python, JavaScript, etc.)}
\cvitem{Frameworks :}{Ex. Spring, React, Angular}
\cvitem{Bases de données :}{Ex. MySQL, PostgreSQL}
\cvitem{Outils :}{Ex. Docker, Git, Jenkins}
\cvitem{Logiciels :}{Ex. Adobe Photoshop, Pack Office}

\sectiontitle{\faIcon{language} LANGUES}
\cvitem{Français :}{ Courant}
\cvitem{Anglais :}{ Professionnel}
\cvitem{Autre langue :}{ Niveau (ex. intermédiaire)}

\end{minipage}%
\hfill%
\begin{tikzpicture}[overlay]
    \draw[color=linecolor] (0,0) -- (0,-25.8);
\end{tikzpicture}%
\hfill%
\begin{minipage}[t]{\rightcolwidth}
\sectiontitle{\faIcon{briefcase} EXPÉRIENCES PROFESSIONNELLES}
\begin{smallerfont}
\cvitem{[Mois Année - Mois Année]}{\textbf{Poste occupé — Nom de l'entreprise}}
Description des responsabilités et tâches :
\begin{itemize}[leftmargin=*,nosep]
\item Réalisation 1
\item Réalisation 2
\item Technologies utilisées : Liste des outils/technologies \\
\end{itemize}

\cvitem{[Mois Année - Mois Année]}{\textbf{Stage/Projet — Nom de l'entreprise}}
Description du projet :
\begin{itemize}[leftmargin=*,nosep]
\item Détail 1
\item Détail 2
\item Technologies utilisées : Liste des outils/technologies
\end{itemize}
\end{smallerfont}

\end{minipage}

\end{document}
